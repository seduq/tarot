\documentclass[12pt]{article}

\usepackage{sbc-template}
\usepackage{graphicx,url}
\usepackage[utf8]{inputenc}
\usepackage[main=portuguese,provide=*]{babel}
\usepackage{amsmath, amssymb, amsthm}
\usepackage{graphicx}
\usepackage{url}
\usepackage{booktabs}
\usepackage{geometry}
\usepackage{setspace}

\newcommand{\argmin}{\operatorname*{arg\,min}}
\newcommand{\argmax}{\operatorname*{arg\,max}}
\DeclareMathOperator{\KL}{KL}



\sloppy

\title{Otimizando IS-MCTS em Jogos de Tarô Francês}

\author{Gustavo F. Ceccon\inst{1} }


\address{
Universidade Estadual Paulista "Júlio de Mesquita Filho"\\
Caixa Postal 13506-692 -- (19) 3526-9000 -- Rio Claro -- SP -- Brasil
\email{gustavo.ceccon@unesp.br}
}

\begin{document} 

\maketitle

\begin{abstract}
This article proposes optimizations for the Information Set Monte Carlo Tree Search (IS-MCTS) algorithm applied to the French Tarot, a card game characterized by high randomness and combinations, which makes uninformed search impossible. The main challenge is the presence of imperfect information, as each player only has a partial view of the game state. The proposal consists of reducing the depth of the search tree through partial and incremental choices, building less detailed but shallower information sets, thus allowing intermediate nodes to be skipped and speeding up the IS-MCTS search process.\end{abstract}
     
\begin{resumo} 
Este artigo propõe otimizações para o algoritmo Information Set Monte Carlo Tree Search (IS-MCTS) aplicadas ao Tarô Francês, um jogo de cartas caracterizado por alta aleatoriedade e combinações que inviabilizam buscas não informadas. O principal desafio abordado é a presença de informação imperfeita, pois cada jogador possui apenas uma visão parcial do estado do jogo. A proposta consiste em reduzir a profundidade da árvore de busca por meio de escolhas parciais e incrementais, construindo conjuntos de informações menos detalhados, porém mais rasos, permitindo assim pular nós intermediários e acelerar o processo de busca do IS-MCTS.
\end{resumo}


\section{Introdução}

\subsection{Contexto e Motivação}

\subsection{Regras do Tarô Francês}

\section{Fundamentação Teórica}

\subsection{Monte Carlo Tree Search (MCTS)}

O \textit{Monte Carlo Tree Search} (MCTS) é um algoritmo de busca heurística amplamente utilizado em jogos e problemas de tomada de decisão sequencial. O MCTS constrói uma árvore de busca de forma incremental, simulando diversas partidas a partir do estado atual do jogo. Cada iteração do algoritmo consiste em quatro etapas principais: seleção, expansão, simulação e retropropagação. Na etapa de seleção, o algoritmo percorre a árvore existente escolhendo os nós mais promissores de acordo com um critério, geralmente baseado no Upper Confidence Bound (UCB). Em seguida, na expansão, um novo nó filho é adicionado à árvore. A simulação realiza uma partida aleatória a partir desse novo nó até um estado terminal. Por fim, na retropropagação, os resultados da simulação são propagados de volta pelos nós visitados, atualizando suas estatísticas. O MCTS é especialmente eficaz em domínios com grandes espaços de estados, pois concentra o esforço computacional nas regiões mais promissoras da árvore de busca.

A fórmula clássica utilizada na etapa de seleção do MCTS é baseada no Upper Confidence Bound for Trees (UCT), dada por:

\begin{equation}
    UCT_i = \frac{w_i}{n_i} + c \cdot \sqrt{\frac{\ln N}{n_i}}
\end{equation}

onde:
\begin{itemize}
    \item $w_i$ é o número de vitórias do nó $i$,
    \item $n_i$ é o número de visitas ao nó $i$,
    \item $N$ é o número total de visitas ao nó pai,
    \item $c$ é um parâmetro de exploração que controla o balanço entre exploração e exploração.
\end{itemize}

\subsection{Information Set Monte Carlo Tree Search (IS-MCTS)}

O \textit{Information Set Monte Carlo Tree Search} (IS-MCTS) é uma extensão do MCTS projetada para jogos com informação imperfeita, nos quais os jogadores não têm acesso completo ao estado do jogo. Em vez de construir a árvore de busca a partir de estados totalmente observáveis, o IS-MCTS utiliza conjuntos de informação, que representam todos os estados possíveis compatíveis com o conhecimento do jogador. A cada simulação, uma determinação (instanciação completa do estado oculto) é amostrada de acordo com as informações disponíveis, e o MCTS é executado sobre essa determinação. Isso permite que o algoritmo avalie ações considerando a incerteza inerente ao ambiente. O IS-MCTS tem sido aplicado com sucesso em jogos de cartas como pôquer e Tarô, onde a modelagem da informação oculta é fundamental para a tomada de decisão eficiente.

\subsection{IS-MCTS no Tarô Francês}
% Intuição ingênua de usar distribuição de cartas e criar nós para cada legal state possível
% Como isso afeta a árvore de busca e do algoritmo
% Complexidade computacional e o uso de memória
% Contratos e declarações e problemas não determinísticos
% Como contrato afeta a pontuação e a escolha de ações

\section{Metodologia}

\subsection{Nó Raiz e Estocasticidade}
% Alpha, Beta e Gamma

\subsection{Information Set Incremental}

\subsection{Profundidade e Largura da Árvore de Busca}

\section{Experimentos e Resultados}

\section{Conclusão e Trabalhos Futuros}

\subsection{Resultados Obtidos}

\subsection{Trabalhos Futuros}

\section{Glossário}

\subsection{Regras do Tarô Francês}

\subsection{Termos Técnicos}


% Figure and table captions should be centered if less than one line
% (Figure~\ref{fig:exampleFig1}), otherwise justified and indented by 0.8cm on
% both margins, as shown in Figure~\ref{fig:exampleFig2}. The caption font must
% be Helvetica, 10 point, boldface, with 6 points of space before and after each
% caption.

% \begin{figure}[ht]
% \centering
% %\includegraphics[width=.5\textwidth]{fig1.jpg}
% \caption{A typical figure}
% \label{fig:exampleFig1}
% \end{figure}

% \begin{figure}[ht]
% \centering
% \includegraphics[width=.3\textwidth]{fig2.jpg}
% \caption{This figure is an example of a figure caption taking more than one
%   line and justified considering margins mentioned in Section~\ref{sec:figs}.}
% \label{fig:exampleFig2}
% \end{figure}


% \begin{table}[ht]
% \centering
% \caption{Variables to be considered on the evaluation of interaction
%   techniques}
% \label{tab:exTable1}
% \includegraphics[width=.7\textwidth]{table.jpg}
% \end{table}

\bibliographystyle{sbc}
\bibliography{article}

\end{document}
